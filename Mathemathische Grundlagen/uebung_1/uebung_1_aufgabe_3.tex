\documentclass{article}
\usepackage[utf8]{inputenc}
\usepackage{amssymb}
\usepackage{amsmath}
\usepackage{parskip}

\title{Mathematische Grundlagen(1141) SoSe 2019 - Übungszettel 1, Aufgabe 3}

\author{Timo Rößner }
\date{März 2019}

\begin{document}

\maketitle

\section*{Aufgabe 3}

Sei

\[
G=
\left\{
  \begin{pmatrix}
    a & b \\
    -b & a
  \end{pmatrix}
| a,b \in M_{22}(\mathbb{R})
\right\}
\]

Beweisen Sie, das die Matrizenmultiplikation eine Verknüpfung auf \(G\) ist, die assoziativ und kommutativ ist und die ein neutrales Element
besitzt. Finden Sie eine Matrix \(A \in G\), für die \(A^2=-I_{2}\) gilt. Dabei ist \(A^2=AA\).

\subsection*{Kommutativität}

Zu zeigen: \(AA'=A'A\)

\[
AA'=
  \begin{pmatrix}
    a & b \\
    -b & a
  \end{pmatrix}
  \begin{pmatrix}
    a' & b' \\
    -b' & a'
  \end{pmatrix}
  =
  \begin{pmatrix}
    aa'-bb'    & ab' + ba' \\
    -ba' - ab' & -bb' + aa'
  \end{pmatrix}
\]

\[
A'A=
  \begin{pmatrix}
    a' & b' \\
    -b' & a'
  \end{pmatrix}
  \begin{pmatrix}
    a & b \\
    -b & a
  \end{pmatrix}
  =
  \begin{pmatrix}
    a'a - b'b  & a'b + b'a \\
    -b'a - a'b & -b'b + a'a
  \end{pmatrix}
  =
  \begin{pmatrix}
    aa' - bb'  & ab' + ba' \\
    -ba' -ab'  & -bb' + aa'
  \end{pmatrix}
\]

Was zu zeigen war.

\subsection*{TODO: Assoziativität:}

Zu zeigen ist, das
\((AA')A''=A(A'A'')\)

%%%%%%%%%%%%%%%%%%%%%%%%%%%%%%
% (AA')A''
%%%%%%%%%%%%%%%%%%%%%%%%%%%%%%

\[
\begin{split}
(AA')A''
=
\left (
\begin{pmatrix}
  a & b \\
  -b & a
\end{pmatrix}
*
\begin{pmatrix}
  a' & b' \\
  -b' & a'
\end{pmatrix}
\right )
*
\begin{pmatrix}
  a'' & b'' \\
  -b'' & a''
\end{pmatrix}
\\
=
\begin{pmatrix}
  aa'-bb'    & ab' + ba' \\
  -ba' - ab' & -bb' + aa'
\end{pmatrix}
*
\begin{pmatrix}
  a'' & b'' \\
  -b'' & a''
\end{pmatrix}
\\
=
% TODO Calculate that.
\end{split}
\]

%%%%%%%%%%%%%%%%%%%%%%%%%%%%%%
% A(A'A'')
%%%%%%%%%%%%%%%%%%%%%%%%%%%%%%

\[
\begin{split}
A(A'A'')
=
\begin{pmatrix}
  a & b \\
  -b & a
\end{pmatrix}
*
\left (
\begin{pmatrix}
  a' & b' \\
  -b' & a'
\end{pmatrix}
*
\begin{pmatrix}
  a'' & b'' \\
  -b'' & a''
\end{pmatrix}
\right )
\\
=
\begin{pmatrix}
  a & b \\
  -b & a
\end{pmatrix}
*
\begin{pmatrix}
  a'a''-b'b'' & a'b'' + b'a'' \\
  -b'a'' - a'b'' & -b'b'' + a'a''
\end{pmatrix}
\\
=
% TODO Calculate this.
\end{split}
\]

\subsection*{Neutrales Element:}

Sei
\(
A=
\begin{pmatrix}
a & b \\
-b & a
\end{pmatrix}
\).

Das neutrale Element \(e\) ist
\(
\begin{pmatrix}
1 & 0 \\
0 & 1
\end{pmatrix}
\)

da gilt:

\[A * e = A\]

\subsection*{Finden Sie eine Matrix \(A \in G\), für die \(A^2=-I_{2}\) gilt:}

Eine Matrix, die obige Bedingung erfüllt ist
\[
A =
\begin{pmatrix}
0 & 1 \\
-1 & 0
\end{pmatrix}
\]

da gilt:

\[
A^2
=
\begin{pmatrix}
0 & 1 \\
-1 & 0
\end{pmatrix}
*
\begin{pmatrix}
0 & 1 \\
-1 & 0
\end{pmatrix}
=
\begin{pmatrix}
-1 & 0 \\
0 & -1
\end{pmatrix}
= -I_{2}
\]

\end{document}

\documentclass{article}
\usepackage[utf8]{inputenc}
\usepackage{parskip}

\title{Mathematische Grundlagen(1141) SoSe 2019 - Übungszettel 1, Aufgabe 1}

\author{Timo Rößner }
\date{März 2019}

\begin{document}

\maketitle

\section*{Aufgabe 1.1}

Seien A, B und C Aussagen. Beweisen Sie, dass folgende Aussagen logisch äquivalent sind.

\section*{Aufgabe 1.1.1}

\(A \wedge (B \lor C)\) und \((A \wedge B) \lor (A \wedge C)\)

\begin{center}
\begin{tabular}{ |c|c|c|c|c|c|c|c| }
 \hline
 A & B & C & \(B \lor C\) & \(A \wedge B\) & \(A \wedge C\) & \(A \wedge (B \lor C)\)& \((A \wedge B) \lor (A \wedge C)\) \\
 \hline\hline
 w & w & w & w & w & w & w & w \\
 w & w & f & w & w & f & w & w \\
 w & f & w & w & f & w & w & w \\
 w & f & f & f & f & f & f & f \\
 f & w & w & w & f & f & f & f \\
 f & w & f & w & f & f & f & f \\
 f & f & w & w & f & f & f & f \\
 f & f & f & f & f & f & f & f \\
 \hline
\end{tabular}
\end{center}

Aus der Gleichheit der vorletzten und der letzten Spalte folgt die Behauptung.

\section*{Aufgabe 1.1.2}

\(A \lor (B \wedge C)\) und \((A \lor B) \wedge (A \lor C)\)

\begin{center}
\begin{tabular}{ |c|c|c|c|c|c|c|c| }
 \hline
 A & B & C & \(B \wedge C\) & \(A \lor B\) & \(A \lor C\) & \(A \lor (B \wedge C)\)& \((A \lor B) \wedge (A \lor C)\) \\
 \hline\hline
 w & w & w & w & w & w & w & w \\
 w & w & f & f & w & w & w & w \\
 w & f & w & f & w & w & w & w \\
 w & f & f & f & w & w & w & w \\
 f & w & w & w & w & w & w & w \\
 f & w & f & f & w & f & f & f \\
 f & f & w & f & f & w & f & f \\
 f & f & f & f & f & f & f & f \\
 \hline
\end{tabular}
\end{center}

Aus der Gleichheit der vorletzten und der letzten Spalte folgt die Behauptung.

\end{document}https://www.overleaf.com/project/5c8e33b08224fd3e79455e90

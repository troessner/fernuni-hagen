\documentclass{article}
\usepackage[utf8]{inputenc}
\usepackage{amssymb}
\usepackage{amsmath}
\usepackage{parskip}

\title{Mathematische Grundlagen(1141) SoSe 2019 - Übungszettel 2, Aufgabe 3}

\author{Timo Rößner }
\date{März 2019}

\begin{document}

\maketitle

\section*{Aufgabe 3}

% https://www.mathelounge.de/561719/ist-ein-erzeugendensystem-der-polynome-mit-hochstens-grad
% https://www.mathelounge.de/77334/erzeugendensystem-raumes-polynome-grad-kleiner-gleich-also
% Lösung via Wolfram Alpha:
%   solve 2a+4b-2c=x, a+b+2c=y, a+c=z for a,b,c

Sei \(V\) der Vektorraum der Polynome vom Grad \(\leq 2\) über \(R\).

Beweisen Sie, das die Polynome \(p_{1} = 2T^2 + T + 1, p_{2} = 4T^2 + T, p_{3} = - 2T^2 + 2T + 1\) ein Erzeugendensystem bilden.

Um dies zu klären, untersuchen wir ob sich jedes Polynom aus \(V\) als Linearkombination von \(p_{1}, p_{2}, p_{3}\) ausdrücken lässt.

\[
\begin{split}
a_{2}T^2 + a_{1}T + a_{0} = \alpha(2T^2 + T + 1) + \beta(4T^2 + T) + \gamma(-2T^2 + 2T + 1) \\
= 2 \alpha T^2 + \alpha T + \alpha + 4\beta T^2 + \beta T + (-2)\gamma T^2 + 2\gamma T + \gamma \\
= 2 \alpha T^2 + 4\beta T^2 -2\gamma T^2 + \alpha T + \beta T + 2\gamma T + \alpha  + \gamma \\
= (2 \alpha + 4\beta -2\gamma) T^2 + (\alpha + \beta + 2\gamma) T + (\alpha  + \gamma)
\end{split}
\]

Damit ergibt sich:

\[
\begin{split}
a_{2} = 2 \alpha + 4 \beta - 2\gamma \\
a_{1} = \alpha + \beta + 2\gamma \\
a_{0} = \alpha + \gamma \\
\end{split}
\]

Nun wenden wir das Gauß Verfahren auf folgende Koeffizientenmatrix an:

\[
\left (
\begin{array}{ccc|c}
  2 & 4 & -2 & a_{2} \\
  1 & 1 & 2 & a_{1} \\
  1 & 0 & 1 & a_{0}
\end{array}
\right )
\]

III - II:

\[
\left (
\begin{array}{ccc|c}
  2 & 4 & -2 & a_{2} \\
  1 & 1 & 2 & a_{1} \\
  0 & -1 & -1 & a_{0} - a_{1}
\end{array}
\right )
\]

2 * II - I:

\[
\left (
\begin{array}{ccc|c}
  2 & 4 & -2 & a_{2} \\
  0 & -2 & 6 & 2a_{1} - a_{2} \\
  0 & -1 & -1 & a_{0} - a_{1}
\end{array}
\right )
\]

2 * III - II:

\[
\left (
\begin{array}{ccc|c}
  2 & 4 & -2 & a_{2} \\
  0 & -2 & 6 & 2a_{1} - a_{2} \\
  0 & 0 & -8 & 2(a_{0} - a_{1}) - (2a_{1} - a_{2})
\end{array}
\right )
\]

Etwas aufräumen ergibt:

\[
\left (
\begin{array}{ccc|c}
  2 & 4 & -2 & a_{2} \\
  0 & -2 & 6 & 2a_{1} - a_{2} \\
  0 & 0 & -8 & 2a_{0} - 4a_{1} + a_{2}
\end{array}
\right )
\]

Die dritte Zeile mal \(\frac{-1}{8}\):

\[
\left (
\begin{array}{ccc|c}
  2 & 4 & -2 & a_{2} \\
  0 & -2 & 6 & 2a_{1} - a_{2} \\
  0 & 0 & 1 & \frac{-1}{4}a_{0} + \frac{1}{2}a_{1} - \frac{1}{8}a_{2}
\end{array}
\right )
\]

Damit gilt:

\[
\gamma = -\frac{1}{4}a_{0} + \frac{1}{2}a_{1} - \frac{1}{8}a_{2}
\]

Einsetzen in die zweite zeile:

\[
-2\beta + 6\gamma = 2a_{1} - a_{2}
\]

liefert:

\[
-2\beta + 6(-\frac{1}{4}a_{0} + \frac{1}{2}a_{1} - \frac{1}{8}a_{2}) = 2a_{1} - a_{2}
\]

Woraus sich nun \(\beta\) ergibt:

\[
-2\beta - \frac{3}{2}a_{0} + 3a_{1} - \frac{3}{4}a_{2} = 2a_{1} - a_{2}
\]

\[
-2\beta = \frac{3}{2}a_{0} - a_{1} - \frac{1}{4}a_{2}
\]

\[
\beta = -\frac{3}{4}a_{0} + \frac{1}{2}a_{1} + \frac{1}{8}a_{2}
\]

Damit haben wir \(\beta\) bestimmt. Nun setzen wir \(\beta\) und \(\gamma\) in die erste Zeile ein:

\[
2\alpha + 4\beta - 2\gamma = a_{2}
\]

\[
2\alpha + 4(-\frac{3}{4}a_{0} + \frac{1}{2}a_{1} + \frac{1}{8}a_{2}) - 2(-\frac{1}{4}a_{0} + \frac{1}{2}a_{1} - \frac{1}{8}a_{2}) = a_{2}
\]

\[
2\alpha -3a_{0} + 2a_{1} + \frac{1}{2}a_{2} + \frac{1}{2}a_{0} - a_{1} + \frac{1}{4}a_{2} = a_{2}
\]

\[
2\alpha - \frac{5}{2}a_{0} + a_{1} + \frac{3}{4}a_{2} = a_{2}
\]

\[
2\alpha = a_{2} + \frac{5}{2}a_{0} - a_{1} - \frac{3}{4}a_{2}
\]

\[
2\alpha = \frac{5}{2}a_{0} - a_{1} + \frac{1}{4}a_{2}
\]

\[
\alpha = \frac{5}{4}a_{0} - \frac{1}{2}a_{1} + \frac{1}{8}a_{2}
\]

\(\alpha\), \(\beta\) und \(\gamma\) sind nun bestimmt als:

\[
\alpha = \frac{5}{4}a_{0} - \frac{1}{2}a_{1} + \frac{1}{8}a_{2}
\]

\[
\beta = -\frac{3}{4}a_{0} + \frac{1}{2}a_{1} + \frac{1}{8}a_{2}
\]

\[
\gamma = -\frac{1}{4}a_{0} + \frac{1}{2}a_{1} - \frac{1}{8}a_{2}
\]

Damit ist gezeigt, das sich jedes Polynom vom Grad \(\leq 2\) als Linearkombination von \(p_{1} = 2T^2 + T + 1, p_{2} = 4T^2 + T, p_{3} = - 2T^2 + 2T + 1\) erzeugen läßt. \(p_{1} = 2T^2 + T + 1, p_{2} = 4T^2 + T, p_{3} = - 2T^2 + 2T + 1\) bilden demzufolge ein Erzeugendensystem bilden.

\textbf{Kurze Stichprobe:}

Versuchen wir das Polynom \(2T^2 + 3T + 4\) dementsprechend zu generieren ergibt sich:

\[
2T^2 + 3T + 4 = (2 \alpha + 4\beta -2\gamma) T^2 + (\alpha + \beta + 2\gamma) T + (\alpha  + \gamma)
\]

Betrachten wir aus Lesbarkeitsgründen nur den ersten Term müsste gelten:

\[
\begin{split}
2 = (2 \alpha + 4\beta -2\gamma) \\
2(\frac{5}{4}a_{0} - \frac{1}{2}a_{1} + \frac{1}{8}a_{2}) + 4(-\frac{3}{4}a_{0} + \frac{1}{2}a_{1} + \frac{1}{8}a_{2}) - 2(-\frac{1}{4}a_{0} + \frac{1}{2}a_{1} - \frac{1}{8}a_{2}) =
\\
\frac{5}{2}a_{0} - a_{1} + \frac{1}{4}a_{2} - 3a_{0} + 2a_{1} + \frac{1}{2}a_{2} + \frac{1}{2}a_{0} - a_{1} + \frac{1}{4}a_{2}) = a_{2}
\end{split}
\]

Der erste Koeffizient unseres Polynoms eingesetzt ergibt:

\[
a_{2} = 2
\]

was wahr ist.

\end{document}

\documentclass{article}
\usepackage[utf8]{inputenc}
\usepackage{amssymb}
\usepackage{amsmath}
\usepackage{parskip}

\title{Mathematische Grundlagen(1141) SoSe 2019 - Übungszettel 2, Aufgabe 2}

\author{Timo Rößner }
\date{März 2019}

\begin{document}

\maketitle

\section*{Aufgabe 2}

Sei
\(
A=
  \begin{pmatrix}
    -1 & 4 & -4 \\
    1 & -3 & 1 \\
    1 & -2 & 0
  \end{pmatrix}
\in M_{33}(\mathbb{R})
\).
Bestimmen Sie in jedem der folgenden Fälle eine Matrix M, sodass \(MA = B\) ist.

Zunächst einmal müssen wir A invertieren da

\[
\begin{split}
MAA^{-1} = BA^{-1} \\
ME = BA^{-1} \\
M = BA^{-1}
\end{split}
\]

\[
\left (
\begin{array}{ccc|ccc}
  -1 & 4 & -4 & 1 & 0 & 0 \\
  1 & -3 & 1 & 0 & 1 & 0 \\
  1 & -2 & 0 & 0 & 0 & 1 \\
\end{array}
\right )
\]

III + I:

\[
\left (
\begin{array}{ccc|ccc}
  -1 & 4 & -4 & 1 & 0 & 0 \\
  1 & -3 & 1 & 0 & 1 & 0 \\
  0 & 2 & -4 & 1 & 0 & 1 \\
\end{array}
\right )
\]

I - III:

\[
\left (
\begin{array}{ccc|ccc}
  -1 & 2 & 0 & 0 & 0 & -1 \\
  1 & -3 & 1 & 0 & 1 & 0 \\
  0 & 2 & -4 & 1 & 0 & 1 \\
\end{array}
\right )
\]

II + I:

\[
\left (
\begin{array}{ccc|ccc}
  -1 & 2 & 0 & 0 & 0 & -1 \\
  0 & -1 & 1 & 0 & 1 & -1 \\
  0 & 2 & -4 & 1 & 0 & 1 \\
\end{array}
\right )
\]

I + 2II:

\[
\left (
\begin{array}{ccc|ccc}
  -1 & 0 & 2 & 0 & 2 & -3 \\
  0 & -1 & 1 & 0 & 1 & -1 \\
  0 & 2 & -4 & 1 & 0 & 1 \\
\end{array}
\right )
\]

III + 2II:

\[
\left (
\begin{array}{ccc|ccc}
  -1 & 0 & 2 & 0 & 2 & -3 \\
  0 & -1 & 1 & 0 & 1 & -1 \\
  0 & 0 & -2 & 1 & 2 & -1 \\
\end{array}
\right )
\]

I + III:

\[
\left (
\begin{array}{ccc|ccc}
  -1 & 0 & 0 & 1 & 4 & -4 \\
  0 & -1 & 1 & 0 & 1 & -1 \\
  0 & 0 & -2 & 1 & 2 & -1 \\
\end{array}
\right )
\]

III * -0.5, II * -1, I * -1:

\[
\left (
\begin{array}{ccc|ccc}
  1 & 0 & 0 & -1 & -4 & 4 \\
  0 & 1 & -1 & 0 & -1 & 1 \\
  0 & 0 & 1 & -0.5 & -1 & 0.5 \\
\end{array}
\right )
\]

II + III:

\[
\left (
\begin{array}{ccc|ccc}
  1 & 0 & 0 & -1 & -4 & 4 \\
  0 & 1 & 0 & -0.5 & -2 & 1.5 \\
  0 & 0 & 1 & -0.5 & -1 & 0.5 \\
\end{array}
\right )
\]

Damit ergibt sich für \(A^{-1}\)
\(
\begin{pmatrix}
  -1 & -4 & 4 \\
  -0.5 & -2 & 1.5 \\
  -0.5 & -1 & 0.5 \\
\end{pmatrix}
\)

Kurze Kontrolle ob die Invertierung korrekt ist:

\[
\begin{pmatrix}
  -1 & 4 & -4 \\
  1 & -3 & 1 \\
  1 & -2 & 0
\end{pmatrix}
*
\begin{pmatrix}
  -1 & -4 & 4 \\
  -0.5 & -2 & 1.5 \\
  -0.5 & -1 & 0.5 \\
\end{pmatrix}
=
\begin{pmatrix}
  1 & 0 & 0 \\
  0 & 1 & 0 \\
  0 & 0 & 1 \\
\end{pmatrix}
\]

Nun zu der Bestimmung von M gegeben ein bestimmtes B:

\[
B =
\begin{pmatrix}
  2 & 1 & 4 \\
  0 & 0 & -7 \\
  4 & 5 & 3 \\
\end{pmatrix}
\]

\[
M = BA^{-1} =
\begin{pmatrix}
  2 & 1 & 4 \\
  0 & 0 & -7 \\
  4 & 5 & 3 \\
\end{pmatrix}
*
\begin{pmatrix}
  -1 & -4 & 4 \\
  -0.5 & -2 & 1.5 \\
  -0.5 & -1 & 0.5 \\
\end{pmatrix}
=
\begin{pmatrix}
  -4.5 & -14 & 11.5 \\
  3.5 & 7 & -3.5 \\
  -8 & -29 & 25 \\
\end{pmatrix}
\]

\[
B =
\begin{pmatrix}
  1 & 2 & 3 \\
  1 & 0 & 1 \\
  2 & 2 & 8 \\
\end{pmatrix}
\]

\[
M = BA^{-1} =
\begin{pmatrix}
  1 & 2 & 3 \\
  1 & 0 & 1 \\
  2 & 2 & 8 \\
\end{pmatrix}
*
\begin{pmatrix}
  -1 & -4 & 4 \\
  -0.5 & -2 & 1.5 \\
  -0.5 & -1 & 0.5 \\
\end{pmatrix}
=
\begin{pmatrix}
  -3.5 & -11 & 8.5 \\
  -1.5 & -5 & 4.5 \\
  -7 & -20 & 15 \\
\end{pmatrix}
\]

\[
B =
\begin{pmatrix}
  0 & 4 & 7 \\
  9 & 3 & 2 \\
  4 & 6 & 6 \\
\end{pmatrix}
\]

\[
M = BA^{-1} =
\begin{pmatrix}
  0 & 4 & 7 \\
  9 & 3 & 2 \\
  4 & 6 & 6 \\
\end{pmatrix}
*
\begin{pmatrix}
  -1 & -4 & 4 \\
  -0.5 & -2 & 1.5 \\
  -0.5 & -1 & 0.5 \\
\end{pmatrix}
=
\begin{pmatrix}
  -5.5 & -15 & 9.5 \\
  -11.5 & -44 & 41.5 \\
  -10 & -34 & 28 \\
\end{pmatrix}
\]

\end{document}

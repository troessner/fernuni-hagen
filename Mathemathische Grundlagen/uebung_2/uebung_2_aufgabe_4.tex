\documentclass{article}
\usepackage[utf8]{inputenc}
\usepackage{amssymb}
\usepackage{amsmath}
\usepackage{parskip}

\title{Mathematische Grundlagen(1141) SoSe 2019 - Übungszettel 2, Aufgabe 4}

\author{Timo Rößner }
\date{März 2019}

\begin{document}

\maketitle

\section*{Aufgabe 4}

Untersuchen Sie, ob
\(
\begin{pmatrix}
    1 \\
    -5 \\
    2
\end{pmatrix}
\in
\left <
\begin{pmatrix}
    1 \\
    -3 \\
    2
\end{pmatrix}
,
\begin{pmatrix}
    2 \\
    -4 \\
    -1
\end{pmatrix}
,
\begin{pmatrix}
    1 \\
    -5 \\
    7
\end{pmatrix}
\right >
\subseteq \mathbb{R^{3}}
\)
gilt.

% Siehe https://www.massmatics.de/merkzettel/#!350:Vektoren_in_linearer_Huelle_enthalten

Zunächst einmal bestimmen wir die Dimension der linearen Hülle in dem wir sie in Matrixform aufstellen und dann ihren Rang berechnen mittel ihrer Zeilenstufenform:

\[
\begin{pmatrix}
    1 & 2 & 1 \\
    -3 & -4 & -5 \\
    2 & -1 & 7
\end{pmatrix}
\]

III - 2*I:

\[
\begin{pmatrix}
    1 & 2 & 1 \\
    -3 & -4 & -5 \\
    0 & -5 & 5
\end{pmatrix}
\]

II + 3*I:

\[
\begin{pmatrix}
    1 & 2 & 1 \\
    0 & 2 & -2 \\
    0 & -5 & 5
\end{pmatrix}
\]

(2 * III) + (5 * II):

\[
\begin{pmatrix}
    1 & 2 & 1 \\
    0 & 2 & -2 \\
    0 & 0 & 0
\end{pmatrix}
\]

Das können wir nicht mehr weiter vereinfachen, und damit ist der Rang der Matrix 2.

Sei nun

\(
V =
\left <
\begin{pmatrix}
    1 \\
    -3 \\
    2
\end{pmatrix}
,
\begin{pmatrix}
    2 \\
    -4 \\
    -1
\end{pmatrix}
,
\begin{pmatrix}
    1 \\
    -5 \\
    7
\end{pmatrix}
,
\begin{pmatrix}
    1 \\
    -5 \\
    2
\end{pmatrix}
\right >
\subseteq \mathbb{R^{4}}
\)

Wieder bestimmen wir die Dimension der linearen Hülle in dem wir sie in Matrixform aufstellen und dann ihren Rang berechnen mittel ihrer Zeilenstufenform:

\[
\begin{pmatrix}
    1 & 2 & 1 & 1 \\
    -3 & -4 & -5 & -5 \\
    2 & -1 & 7 & 2
\end{pmatrix}
\]

III - 2*I:

\[
\begin{pmatrix}
    1 & 2 & 1 & 1 \\
    -3 & -4 & -5 & -5 \\
    0 & -5 & 5 & 0
\end{pmatrix}
\]

II + 3*I:

\[
\begin{pmatrix}
    1 & 2 & 1 & 1 \\
    0 & 2 & -2 & -2 \\
    0 & -5 & 5 & 0
\end{pmatrix}
\]

(2 * III) + (5 * II):

\[
\begin{pmatrix}
    1 & 2 & 1 & 1 \\
    0 & 2 & -2 & -2 \\
    0 & 0 & 0 & -10
\end{pmatrix}
\]

Dies läßt sich nicht weiter vereinfachen und der Rang der Matrix ist 3.

Damit gilt

\(
\begin{pmatrix}
    1 \\
    -5 \\
    2
\end{pmatrix}
\not\in
\left <
\begin{pmatrix}
    1 \\
    -3 \\
    2
\end{pmatrix}
,
\begin{pmatrix}
    2 \\
    -4 \\
    -1
\end{pmatrix}
,
\begin{pmatrix}
    1 \\
    -5 \\
    7
\end{pmatrix}
\right >
\)

\end{document}

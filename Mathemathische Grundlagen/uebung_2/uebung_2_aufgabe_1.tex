\documentclass{article}
\usepackage[utf8]{inputenc}
\usepackage{amssymb}
\usepackage{amsmath}
\usepackage{parskip}

\title{Mathematische Grundlagen(1141) SoSe 2019 - Übungszettel 2, Aufgabe 1}

\author{Timo Rößner }
\date{März 2019}

\begin{document}

\maketitle

\section*{Aufgabe 1}

Bestimmen Sie die Lösungsmenge des folgenden linearen Gleichungssystems über \(R\).

\[
\begin{split}
x1 + 2x2 − 3x3 − 2x4 + 4x5 = 1 \\
2x1 + 5x2 − 8x3 − x4 + 6x5 = 4 \\
x1 + 4x2 − 7x3 + 5x4 + 2x5 = 8.
\end{split}
\]

Im ersten Schritt schreiben wir obiges LGS als Koeffizientenmatrix und wenden dann das Gaußsche Lösungsverfahren an um zur Treppennormalform zu gelangen:

\[
\begin{split}
\begin{pmatrix}
  1 & 2 & -3 & -2 & 4 \\
  2 & 5 & -8 & -1 & 6 \\
  1 & 4 & -7 & 5 & 2 \\
\end{pmatrix}
\end{split}
\]

\end{document}

\documentclass{article}
\usepackage[utf8]{inputenc}
\usepackage{amssymb}
\usepackage{amsmath}
\usepackage{parskip}

\title{Mathematische Grundlagen(1141) SoSe 2019 - Übungszettel 2, Aufgabe 1}

\author{Timo Rößner }
\date{März 2019}

\begin{document}

\maketitle

\section*{Aufgabe 1}

Bestimmen Sie die Lösungsmenge des folgenden linearen Gleichungssystems über \(R\).

\[
\begin{split}
x_{1} + 2x_{2} - 3x_{3} - 2x_{4} + 4x_{5} = 1 \\
2x_{1} + 5x_{2} - 8x_{3} - x_{4} + 6x_{5} = 4 \\
x_{1} + 4x_{2} - 7x_{3} + 5x_{4} + 2x_{5} = 8.
\end{split}
\]

Im ersten Schritt schreiben wir obiges LGS als Koeffizientenmatrix und wenden dann das Gaußsche Lösungsverfahren an um zur Treppennormalform zu gelangen:

\[
\left (
\begin{array}{ccccc|c}
  1 & 2 & -3 & -2 & 4 & 1 \\
  2 & 5 & -8 & -1 & 6 & 4 \\
  1 & 4 & -7 & 5 & 2 & 8 \\
\end{array}
\right )
\]

III - I:

\[
\left (
\begin{array}{ccccc|c}
  1 & 2 & -3 & -2 & 4 & 1 \\
  2 & 5 & -8 & -1 & 6 & 4 \\
  0 & 2 & -4 & 7 & -2 & 7 \\
\end{array}
\right )
\]

II - 2*I:

\[
\left (
\begin{array}{ccccc|c}
  1 & 2 & -3 & -2 & 4 & 1 \\
  0 & 1 & -2 & 3 & -2 & 2 \\
  0 & 2 & -4 & 7 & -2 & 7 \\
\end{array}
\right )
\]

III - 2*II:

\[
\left (
\begin{array}{ccccc|c}
  1 & 2 & -3 & -2 & 4 & 1 \\
  0 & 1 & -2 & 3 & -2 & 2 \\
  0 & 0 & 0 & 1 & 2 & 3 \\
\end{array}
\right )
\]

Damit ist die Matrix nun in Zeilenstufenform. Die letzte Zeile betrachtend gilt:

\[
x_{4} + 2x_{5} = 3
\]

Setzen wir nun \(x_{5} = r\) haben wir

\[
\begin{split}
x_{4} + 2r = 3 \\
x_{4} = 3 - 2r
\end{split}
\]

Für die zweite Zeile heißt das

\[
\begin{split}
x_{2} - 2x_{3} + 3x_{4} - 2x_{5} = 2 \\
x_{2} - 2x_{3} + 3(3 - 2r) - 2r = 2 \\
x_{2} - 2x_{3} + 9 - 6r - 2r = 2 \\
x_{2} - 2x_{3} + 9 - 8r = 2 \\
\end{split}
\]

Setzen wir nun \(x_{3} = s\) haben wir

\[
\begin{split}
x_{2} - 2s + 9 - 8r = 2 \\
x_{2} = 2s + 8r - 7 \\
\end{split}
\]

Für die erste Zeile heißt das

\[
\begin{split}
x_{1} + 2x_{2} - 3x_{3} - 2x_{4} + 4x_{5} = 1 \\
x_{1} + 2(2s + 8r - 7) - 3s - 2(3 - 2r) + 4r = 1 \\
x_{1} + 4s + 16r - 14 - 3s - 6 + 4r + 4r = 1 \\
x_{1} = 1 + 6 + 14 - 4r - 4r - 16r - 4s + 3s \\
x_{1} = 21 - 24r - s \\
\end{split}
\]

Die Lösungsmenge des LGS ist damit

\[
L = \{21 - 24r - s, 2s + 8r - 7, s, 3 - 2r, r\}
\]

Setzen wir \(r = 1, s = 1\) erhalten wir \(\{-4, 3, 1, 1, 1\}\)

Eingesetzt in unser ursprüngliches LGS

\[
\begin{split}
x_{1} + 2x_{2} - 3x_{3} - 2x_{4} + 4x_{5} = 1 \\
2x_{1} + 5x_{2} - 8x_{3} - x_{4} + 6x_{5} = 4 \\
x_{1} + 4x_{2} - 7x_{3} + 5x_{4} + 2x_{5} = 8.
\end{split}
\]

erhalten wir

\[
\begin{split}
-4 + 6 - 3 - 2 + 4 = 1 \\
-8 + 15 - 8 - 1 + 6 = 4 \\
-4 + 12 - 7 + 5 + 2 = 8.
\end{split}
\]

\end{document}

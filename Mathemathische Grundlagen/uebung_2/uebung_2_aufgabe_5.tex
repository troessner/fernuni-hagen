\documentclass{article}
\usepackage[utf8]{inputenc}
\usepackage{amssymb}
\usepackage{amsmath}
\usepackage{parskip}

\title{Mathematische Grundlagen(1141) SoSe 2019 - Übungszettel 2, Aufgabe 5}

\author{Timo Rößner }
\date{März 2019}

\begin{document}

\maketitle

\section*{Aufgabe 5}

Sei \(V\) der Vektorraum \(n x n\)-Matrizen über einem Körper \(\mathbb{K}\). Untersuchen Sie, ob die folgenden Teilmengen \(U\) und \(W\) von \(V\) Unterräume von \(V\) sind.

% Siehe https://de.wikibooks.org/wiki/Mathe_f%C3%BCr_Nicht-Freaks:_Untervektorraum

\subsection*{U als Unterraum}

\[
U =
\left \{
A = (a_{ij}) \in V | a_{ij} = a_{ji} \forall 1 \leq i,j \leq n
\right \}
\]

Zu zeigen: \(U\) ist ein Unterraum von \(V\):

1.) Die Nullmatrix muss in \(U\) enthalten sein.

Dies ist offensichtlich der Fall, da auch für die Nullmatrix \( a_{ij} = a_{ji} \forall 1 \leq i,j \leq n \) erfüllt ist.

2.) \(U\) ist abgeschlossen unter der Skalarmultplikation.

Für alle \(v \in U\) und \(\lambda \in \mathbb{R}\) ist \(\lambda v \in U\) da \(\lambda a_{ij} = \lambda a_{ji}\).

3.) \(U\) ist abgeschlossen unter der Addition.

Seien \(A, B \in U\) und \(C = A + B\) dann gilt für jeden Eintrag in \(C\) \(c_{ij} = a_{ij} + b_{ij}\) und für \(c_{ji} = a_{ji} + b_{ji}\) womit auch diese Bedingung erfüllt wäre.

Damit ist \(U\) ein Unterraum von \(V\).

\subsection*{V als Unterraum}

Sei \(T\) eine feste Matrix in \(V\) und sei
\(W =
\left \{
A \in V | AT = TA
\right \}
\)

Aus der Aufgabenstellung wird klar, das entweder \(A\) oder \(T\) die Einheitsmatrix sein muss. Definieren wir im Folgenden \(T\) als die Einheitsmatrix.

1.) Die Nullmatrix muss in \(V\) enthalten sein.

Dies ist erfüllt wenn \(A\) selbst die Nullmatrix ist.

2.) \(V\) ist abgeschlossen unter der Skalarmultplikation.

\(V\) ist abgeschlossen unter der Skalarmultplikation da \(\lambda(AT) = \lambda(A) = \lambda(TA)\) ist.

3.) \(V\) ist abgeschlossen unter der Addition.

\(V\) ist abgeschlossen unter der Addition da \((AT) + C = A + C = (TA) + C\) ist.

Damit ist \(V\) ein Unterraum von \(V\).

\end{document}

\documentclass{article}
\usepackage[utf8]{inputenc}
\usepackage{parskip}

\title{Mathematische Grundlagen(1141) SoSe 2019 - Übungszettel 1, Aufgabe 5}

\author{Timo Rößner }
\date{März 2019}

\begin{document}

\maketitle

\section*{Aufgabe 1.5}

Sei \(n \in \mathbb{N}\). Eine binäre Folge der Länge \(n\) ist ein Ausdruck der Form \((x_{1}, ..., x_{n})\), wobei \(x_{i} \in \{0, 1\}\) für alle \(1 \leq i \leq n\).

\subsection*{Bestimmen Sie alle binären Folgen der Länge 1,2,3.}

Für \(n = 1\):

(0), (1)

Für \(n = 2\):

(00), (01), (10), (11)

Für \(n = 3\):

(000), (001), (010), (011), (100), (101), (110), (111)

\subsection*{ Finden Sie eine Formel für die Anzahl der binären Folgen \(n\), wobei \(n\) eine feste, natürliche Zahl ist, und beweisen Sie diese Formel mit Induktion nach \(n\).}

Für \(n = 1\) hat man zwei Möglichkeiten, entweder 0 oder 1. Für \(n = 2\) sind es 4 Möglichkeiten, für \(n = 3\) 8 Möglichkeiten.

Sei \(a_{n}\) eine binäre Folge und \(|a_{n}|\) die Anzahl der möglchen Folgen, so steht zu vermuten, das \(|a_{n}| = 2^{n}\).

\textbf{Behauptung}:

Für alle \(n \in \mathbb{N}\) ist \(|a_{n}| = 2^{n}\)

\textbf{Beweis:}

\textbf{Induktionsanfang:} Sei \(n = 1\) dann ist \(|a_{1}| = 2^{1} = 2\) was wahr ist (siehe oben).

\textbf{Induktionsannahme:} Sei \(n > 1\) dann ist \(|a_{n}| = 2^{n}\)

\textbf{Induktionsschritt:} Zu zeigen ist, das \(|a_{n+1}| = 2^{n+1}\)

Da das Hinzufügen eines weiteren Folgengliedes die Anzahl der Möglichkeiten der Folge verdoppelt gilt:

\[
|a_{n+1}| = |a_{n}| * 2
\]

Mit der Induktionsannahme gilt nun:

\[
|a_{n+1}| = |a_{n}| * 2 = 2^{n} * 2 = 2^{n+1}
\]

was zu zeigen war.



\end{document}

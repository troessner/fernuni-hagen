\documentclass{article}
\usepackage[utf8]{inputenc}
\usepackage{amssymb}
\usepackage{amsmath}
\usepackage{parskip}

\title{Mathematische Grundlagen(1141) SoSe 2019 - Übungszettel 1, Aufgabe 2}

\author{Timo Rößner }
\date{März 2019}

\begin{document}

\maketitle

\section*{Aufgabe 2}

Sei \(\mathbb{K}\) ein Körper und \(f: M_{22}(\mathbb{K}) \rightarrow \mathbb{K}\) definiert durch
\(
M=
  \begin{pmatrix}
    a & b \\
    c & d
  \end{pmatrix}
\rightarrow ad - bc
\)
für alle
\(
M=
  \begin{pmatrix}
    a & b \\
    c & d
  \end{pmatrix}
\in M_{22}(\mathbb{K})
\)

Beweisen oder widerlegen Sie folgende Behauptungen:

\section{TODO: Die Abbildung f ist surjektiv:}


\section{Die Abbildung f ist nicht injektiv.}

Sei

\[
A=
  \begin{pmatrix}
    1 & 2 \\
    3 & 4
  \end{pmatrix}
,
B=
  \begin{pmatrix}
    4 & 3 \\
    2 & 1
  \end{pmatrix}
\]

dann ist

\[f(A) = 1*4 - 2*3 = f(B) = 4*1 - 3*2\]

womit wir ein Element im Bild von \(f\) gefunden haben mit mehr als einem Urbild. Damit ist \(f\) nicht injektiv.

\section{TODO: Für alle \(A, B\in M_{22}(\mathbb{K})\) gilt \(f(AB) = f(A) f(B)\):}

Sei

\[
A=
  \begin{pmatrix}
    a & b \\
    c & d
  \end{pmatrix}
,
B=
  \begin{pmatrix}
    a' & b' \\
    c' & d'
  \end{pmatrix}
\]

dann gilt

\[
f(A)f(B)=(ad - bc)(a'd' - b'c') = aa'dd' - adb'c' - bca'd' + bb'cc'
= ada'd' - adb'c' - bca'd' + bcb'c'
\]

und

\[
AB=
  \begin{pmatrix}
    a & b \\
    c & d
  \end{pmatrix}
  \begin{pmatrix}
    a' & b' \\
    c' & d'
  \end{pmatrix}
  =
  \begin{pmatrix}
    aa'bc' & ab'bd' \\
    ca'dc' & cb'dd'
  \end{pmatrix}
  =
  \begin{pmatrix}
    aba'c' & abb'd' \\
    cda'c' & cdb'd'
  \end{pmatrix}
\]

und

\[
f(AB)= aba'c'cdb'd' - abb'd'cda'c'
\]

\section{TODO: Für alle \(A, B\in M_{22}(\mathbb{K})\) gilt \(f(A+B) = f(A) + f(B)\)}

\end{document}
